% !TeX document-id = {32a890c7-0db2-461b-b834-605db552d4de}
% !TEX root
% !TEX program = xelatex
% !BIB program = biber

%\def \PrintMode{} % 在使用电子版论文时(oneside),请将此行注释。
				  % 在打印纸质论文时(twoside),请保持本行命令不被注释,然后打印时选择双面打印即可。


%用来控制是否启动打印模式的宏,请勿改动。
\ifx \PrintMode \undefined
    \def \SideMode{oneside}
    \def \ClearPageStyle{\clearpage}
\else
    \def \SideMode{twoside}
    \def \ClearPageStyle{\cleardoublepage}
\fi

\documentclass[a4paper,\SideMode,UTF8]{book} %A4纸,UTF-8

\input{packages_and_settings.tex} %加载各宏包以及本模板的主要设置
\addbibresource{reference/refs.bib} %加载bib文件(参考文献)
%将参考文献字体设置为五号
\renewcommand*{\bibfont}{\zihao{5}}
%\setlength{\bibsep}{1pt plus 0.3ex}

\newcommand{\citeay}[1]{(\citeauthor{#1},\citeyear{#1})\cite{#1}}  % (姓名,年)【#】
\newcommand{\citeayp}[1]{\citeauthor{#1}(\citeyear{#1})\cite{#1}}  %  姓名(年)【#】

\DeclareDelimFormat[citeauthor]{finalnamedelim}{ \& }
\DeclareDelimFormat{finalnamedelim}{\addspace and \space}



\graphicspath{{figures/}}   % 设置图片所存放的目录

\begin{document}


\newcommand{\TitleCHS}{论文题目} %中文标题

\newcommand{\TitleENG}{\LaTeX\xspace Template for  Undergraduate Dissertation in ECNU } %英文标题

\newcommand{\Author}{~高家灿, 何茜彤, 周梓桐~} %作者名字

\newcommand{\StudentID}{~51274404-081/016/117~} %学号

\newcommand{\Department}{~统计学院~} %学院

\newcommand{\Major}{~应用统计~} %专业

\newcommand{\Supervisor}{~张思亮~} %导师名字

\newcommand{\AcademicTitle}{~副教授~} %导师职称

\newcommand{\CompleteDate}{~2025年11月~} %毕业年份

\newcommand{\CompleteYear}{2025} %毕业月份

\newcommand{\CompleteMonth}{5} %毕业月份

\newcommand{\KeywordsCHS}{关键词1,关键词2,关键词3,关键词4,关键词5,关键词6,关键词7 } %中文关键词

\newcommand{\KeywordsENG}{keyword1, keyword2, keyword3, keyword4,keyword5, keyword6, keyword7} %英文关键词

 
\pagestyle{empty} %不对正文前的各页面使用页眉页脚
\newgeometry{top=2.0cm, bottom=2.0cm,left=3.18cm, right=3.18cm} %设置用于首页的页边距
\input{preface/inner-cover.tex} %插入内封面
\ClearPageStyle

\restoregeometry
%生成目录
\addtocontents{toc}{\protect\thispagestyle{empty}}
\begin{spacing}{1}
    \tableofcontents
\end{spacing}
\ClearPageStyle
\pagenumbering{Roman}

\input{preface/abstract-CHS.tex}   %生成中文摘要及关键词
\ClearPageStyle

\input{preface/abstract-ENG.tex}  %生成英文摘要及关键词
\ClearPageStyle

\pagenumbering{arabic}
\pagestyle{fancy} %开始使用页眉页脚
\setcounter{page}{1} %论文页码从正文开始记数

\chapter{引言}\label{chap1}

\section{研究背景}

简单介绍与论文选题有关的背景资料,包括国内外的研究现状,存在的问题,主要的参考文献,研究本文的动机,以后部分论文的基本结构。

\section{研究框架}

\begin{enumerate}
	\def\labelenumi{\arabic{enumi}.}
	\item
	根据华东师范大学本科毕业论文的要求定制(使用\TeX{}技术)
	\item
	相比于Word和TeX提升50-80\%的工作效率
	\item
	通过Rmarkdown包实现对R, markdown, \TeX{}的全面支持
	\item
	标准格式的pdf输出
	\item
	标准的高精度\TeX{}输出
	\item
	支持\TeX{}语法
	\item
	通过章节分类管理实现快速编译与整合
	\item
	支持直接运行R和Python代码,并将生成的图形和表格嵌入到文档中
	\item
	支持本地图形的插入
	\item
	支持生成的R与Python图形自动添加题注(caption)
	\item
	支持使用\TeX{}命令对浮动公式、图形和表格进行引入
	\item
	支持R代码抄录,且语法高亮显示
	\item
	支持Python代码抄录,且语法高亮显示
	\item
	免去复杂\TeX{}命令, 仅通过简单的markdown标记语言实现快速写作
\end{enumerate}

\section{创新点}
 %正文第一章
\chapter{数据与方法概述}\label{chap2}

\section{数据来源与变量说明}

毕业论文格式应规范,必须由封面、目录、正文(包括中外文题名、中外文摘要、中外文关键词、正文、参考文献和致谢)三部分构成。论文装订顺序为

\begin{itemize}
\item
外封面
\item
开题报告
\item
内封面
\item
目录
\item
中文摘要: 中文题名,中文摘要内容,中文关键词
\item
英文摘要: 英文题名, 英文摘要内容, 英文关键词
\item
正文
\item
参考文献(至少包含二篇英文文献)
\item
附录
\item
致谢
\item
考核意见表
\end{itemize}

\section{数据预处理}
这节用来展示文章的5层结构。事实上,一般来说文章层次在3-4层为宜。在之后的section中,我们会只使用至多3层结构(即,节-小节-子节)来进行各种演示。
 
\subsection{子节标题}这一子节我们介绍这些内容。

\subsubsection{子子节标题}这一段我们详细介绍这些内容。 

\paragraph{段标题}这一段我们介绍这些内容。 

%\subparagraph{小段标题}这一小段我们介绍这些内容。

\section{统计推断与结构发现} %正文第二章
\chapter{特征工程与监督学习实验设计}\label{chap3}

\section{A/B 特征工程:Full-8D vs PCA-2D}

\subsection{字数要求}

5000字以上. 

\subsection{封面要求}

\subsubsection{要求1}  
上交的每份论文都一律采用学校统一印发的外封面(装订线一律在左面)。

\subsubsection{要求2}
另附自制内封面一份(A4纸张电脑打印),内容为中外文论文题目、作者的姓名、学号、班级、指导老师的姓名与职称、论文完成时间。

\subsection{开题报告要求}
开题报告内容包括:选题的背景与意义(对与选题有关的国内外研究现状、进展情况、存在的问题等进行调研,在此基础上提出选题的研究意义),课题研究的主要内容、方法、技术路线,课题研究拟解决的主要问题及创新之处,课题研究的总体安排与进度,参考文献等方面。开题报告表格至教务处网站下载。

\section{监督学习实验设计}


\subsection{有序列表}
\begin{enumerate}
	\item
	项目列表
	\item
	项目列表
	\item
	项目列表
	
	\begin{enumerate}
		\item
		项目列表
		\item
		项目列表
		\item
		项目列表
	\end{enumerate}
\end{enumerate}

\subsection{无序列表}


\begin{itemize}
	\item
	项目列表
	\item
	项目列表
	
	\begin{itemize}
		\item
		项目列表
		\item
		项目列表
		\item
		项目列表
	\end{itemize}
\end{itemize}

\subsection{公式}
\subsubsection{单行公式}
\begin{itemize}
	\item
	公式示例1: \[\mu_1\le\mu_2\le \dots\le \mu_k.\]
	\item
	公式示例2: 
	\begin{equation}\label{eq1}
	x^2+y^2=1
	\end{equation}
	\item
	引用: 公式\eqref{eq1}或使用cleveref包,\cref{eq1}.
\end{itemize}

\subsubsection{多行公式}

\begin{itemize}
	\item
	公式示例3:\\
	\begin{align}
	x^2+y^2 &= 1 \label{eq4}\\
	x_2+y_2 &= 0  \label{eq5}
	\end{align}
	\item
	引用: \cref{eq4}和\cref{eq5}
	\item
	公式示例4:\\
	\[
	f(x)=\begin{cases} 
		1, & \text{If $x\ge 0$}, \\ 
		0, & \text{Otherwise,}
	\end{cases}
	\]
	\item
	公式示例5:\\
	
	\begin{numcases}{|x|=}
	x, & for $x \geq 0$\\
	-x, & for $x < 0$
	\end{numcases}
\end{itemize}

\subsection{align环境}
\begin{align*}
\operatorname{E} (Z_{n+1} - Z_n | X_1,..., X_n)
&= \operatorname{E} (S_{n+1}^2 - (n+1) \sigma^2 - S_n^2 + n \sigma^2 | X_1,..., X_n) \\
&= \operatorname{E} (S_{n+1}^2 - S_n^2 - (n+1) \sigma^2 + n \sigma^2 | X_1,..., X_n) \\
&= \operatorname{E} (X_{n+1}(X_{n+1} + 2\sum_{i=1}^n X_i) - \sigma^2 | X_1,..., X_n) \\
&= \operatorname{E} (X_{n+1}X_{n+1})
+ 2\operatorname{E} (X_{n+1}) \sum_{i=1}^n X_i - \sigma^2 \\
&= \sigma^2  - \sigma^2 =0.
\end{align*}

\subsection{split环境(内嵌)}
\begin{equation*}
\begin{split}
(a + b)^4
&= (a + b)^2 (a + b)^2      \\
&= (a^2 + 2ab + b^2)
(a^2 + 2ab + b^2)        \\
&= a^4 + 4a^3b + 6a^2b^2 + 4ab^3 + b^5
\end{split}
\end{equation*}

\subsection{带大括号的多行公式}
\paragraph{cases}
$$
f=
\begin{cases}
x + y = z,  \\
1 + 2 = 3.  \\
\end{cases}
$$

\paragraph{array}
$$ F^{HLLC}=\left\{
\begin{array}{rcl}
F_L       &      & {0      <      S_L}\\
F^*_L     &      & {S_L \leq 0 < S_M}\\
F^*_R     &      & {S_M \leq 0 < S_R}\\
F_R       &      & {S_R \leq 0}
\end{array} \right. $$

\paragraph{aligned}
\begin{equation}
\left\{
\begin{aligned}
\overset{.}x(t) &=A_{ci}x(t)+B_{1ci}w(t)+B_{2ci}u(t)  \\
z(t) &=C_{ci}x(t)+D_{ci}u(t) \\
\end{aligned}
\right.
\end{equation}


\subsection{表格}
本来\LaTeX 里表格的变化是非常多的,但鉴于学校要求用三线式,问题反而简单了, 见下面的例子:\cref{tab:1}和\cref{tab:1}. 
\begin{table}[htbp]\center
	\caption{\label{tab:1}示例表格\\Table \ref{tab:1}\quad Example Table}
	\begin{tabular}{lcccccl}
		\toprule
		。。 & 。。 & 。。 & 。。 & 。。& 。。 & 。。\\
		\midrule
		。。 & 。。 & 。。 & 。。 & 。。& 。。 & 。。\\
		。。 & 。。 & 。。 & 。。 & 。。& 。。 & 。。\\
		。。 & 。。 & 。。 & 。。 & 。。& 。。 & 。。\\
		。。 & 。。 & 。。 & 。。 & 。。& 。。 & 。。\\
		。。 & 。。 & 。。 & 。。 & 。。& 。。 & 。。\\
		\bottomrule
	\end{tabular}
\end{table}


\begin{table}[ht]
	\centering
	\caption{\label{tab:2}Iris数据\\Table \ref{tab:2}\quad Iris Data.} 
	\begin{tabular}{rrrrrl}
		\hline
		& Sepal.Length & Sepal.Width & Petal.Length & Petal.Width & Species \\ 
		\hline
		1 & 5.10 & 3.50 & 1.40 & 0.20 & setosa \\ 
		2 & 4.90 & 3.00 & 1.40 & 0.20 & setosa \\ 
		3 & 4.70 & 3.20 & 1.30 & 0.20 & setosa \\ 
		4 & 4.60 & 3.10 & 1.50 & 0.20 & setosa \\ 
		5 & 5.00 & 3.60 & 1.40 & 0.20 & setosa \\ 
		6 & 5.40 & 3.90 & 1.70 & 0.40 & setosa \\ 
		\hline
	\end{tabular}
\end{table}


\subsection{插图}
由于这份模板不考虑多栏排版, 以下是二个通栏图的演示:\cref{fig:1}和\cref{fig:2}
\begin{figure}[H]
	\centering
	\includegraphics[width=0.7\textwidth]{example-image}
	\caption{图片测试(最小宽度)\\Figure \ref{fig:1}\quad  Image test (Minimal width)\label{fig:1}}
\end{figure}

\begin{figure}[H]
	\centering
	\includegraphics[width=130mm]{example-image}
	%\includegraphics[width=130mm]{./figures/你自己的图像文件}
	\caption{图片测试(最大宽度)\\Figure \ref{fig:2}\quad  Image test (Maximal width)\label{fig:2}}
\end{figure}

注意:这里为了减少图片上下的空白,使用了float宏包。


\newcommand{\bbb}{这是一个针对定理类环境进行的科技文稿排版测试}

\subsection{定理型环境示例}

\begin{definition} 
	这是一个针对定理类环境进行的科技文稿排版测试
\end{definition}

\begin{theorem}\label{th1}
	这是一个针对定理类环境进行的科技文稿排版测试
\end{theorem}

\begin{proof}
	这是一个针对定理类环境进行的科技文稿排版测试
\end{proof}

\begin{corollary}\label{cor1}
	这是一个针对定理类环境进行的科技文稿排版测试
\end{corollary}

\begin{lemma}\label{lem1}
	这是一个针对定理类环境进行的科技文稿排版测试
\end{lemma}

\begin{example}
	这是一个针对定理类环境进行的科技文稿排版测试
\end{example}

\subsection{脚注与引用}

\subsubsection{脚注}

这里是脚注测试\footnote{1111111111}这里是脚注测试这里是脚注测试这里是脚注测试\footnote{2222222222}这里是脚注测试这里是脚注测试这里是脚注测试这里是脚注测试这里是脚注测试这里是脚注测试这里是脚注测试这里是脚注测试这里是脚注测试这里是脚注测试这里是脚注测试这里是脚注测试这里是脚注测试这里是脚注测试这里是脚注测试\footnote{3333333333}这里是脚注测试这里是脚注测试这里是脚注测试这里是脚注测试这里是脚注测试这里是脚注测试这里是脚注测试这里是脚注测试这里是脚注测试这里是脚注测试这里是脚注测试这里是脚注测试

\subsubsection{定理类引用}

由定理\ref{th1}我们可以知道XXXXXXXX。

由引理\ref{lem1}我们可以知道XXXXXXXX。

由推论\ref{cor1}我们可以知道XXXXXXXX。

\subsubsection{文献引用的演示}

本模板使用biblatex进行文献管理,这是一套相对较新的系统。另外,使用了hushidong制作的符合gb7714-2015标准的biblatex样式。在此对他的工作表示感谢,要完成这样的样式非常不容易。本模板中gb7714-2015.bbx与gb7714-2015.cbx即为他的作品,在这里打包发布以便使用。详见\url{https://github.com/hushidong/biblatex-gb7714-2015}查找相关资料。

默认的bib文件位于\textasciitilde{}/reference/thesis-ref.bib,内容是由Wang
Tianshu制作,在此仅作演示之用。关于bib文件的编写与管理请自行查找相关教程。

默认的bib文件位于~/reference/thesis-ref.bib,内容是由Wang Tianshu制作,在此仅作演示之用。关于bib文件的编写与管理请自行查找相关教程。

下方的演示已经给出了正文中引用文献的基本方法,这与传统的cite命令是类似的。如有更多需求,请至\url{https://github.com/hushidong/biblatex-gb7714-2015}查找相关资料。

文献\parencite{Wuwei:2013}中提到xxxxxxx\citeay{tang:2008}.

文献\parencite{zhouxu:2019}中提到yyyyyyy。

\citeayp{tang:2008}提到zzzzzzz。

\textcolor{blue}{本模板使用biblatex宏包处理文献库bib,
			建议使用自定义的$\backslash$citeay\{\}和$\backslash$citeayp\{\}命令。}


\nocite{*}
 %正文第三章
\input{body/ch4-supervised.tex} %正文第四章
\input{body/ch5-latent.tex} %正文第五章
\input{body/ch6-clustering.tex} %正文第六章
\input{body/ch7-conclusion.tex} %正文第七章

%生成参考文献
\phantomsection
\setlength{\bibitemsep}{0pt}
\addcontentsline{toc}{chapter}{参考文献}
\printbibliography[title={\centerline{\bfseries\sffamily \zihao {-3}参考文献}}]
\ClearPageStyle


%生成感谢
\makeacknowledgement 
\ClearPageStyle




%生成附录
\phantomsection
\begin{appendix}
	\renewcommand{\chaptername}{附录 \Alph{chapter}}
	\renewcommand{\thesection}{\Alph{chapter}.\arabic{section}}
	\renewcommand{\thesubsection}{\Alph{chapter}.\arabic{section}.\arabic{subsection}}
	\renewcommand{\thesubsubsection}{\arabic{subsubsection}.}
	\renewcommand{\thetable}{\Alph{chapter}-\arabic{table}}
	\renewcommand{\theequation}{\Alph{chapter}-\arabic{equation}}
	\renewcommand{\thefigure}{\Alph{chapter}-\arabic{figure}}

\addtocontents{toc}{\setcounter{tocdepth}{1}}
\setcounter{subsection}{0}

%\ctexset { subsection = { name={,},number={\arabic{subsection}},format={\rmfamily \zihao {5}} } }
%\ctexset { subparagraph = { name={(,)},number={\arabic{subparagraph}},format={\rmfamily \zihao {5}},indent=2em } }


\chapter{附录标题}
\section{附录中的图形、表格、公式}

\subsection{公式}
附录中的公式(\ref{equ:Call})和(\ref{equ:Put})分别为:
\begin{equation}\label{equ:Call}
c=S_0N(d_1)-X e^{-r T}N(d_2)
\end{equation}
和
\begin{equation}\label{equ:Put}
p=X e^{-r T}N(-d_2)-S_0N(-d_1),
\end{equation}

\subsection{图形}
\begin{figure}[htbp!]
	\centering
	\includegraphics[width=100mm]{example-image}
	\caption{图片测试(最小宽度)\\Figure \ref{fig-a1}\quad  Image test (Minimal width)\label{fig-a1}}
\end{figure}

\subsection{表格}
\begin{table}[ht]
	\centering
	\caption{\label{tab-aa}Iris数据} 
	\begin{tabular}{rrrrrl}
		\toprule
		& Sepal.Length & Sepal.Width & Petal.Length & Petal.Width & Species \\ 
		\midrule
		1 & 5.10 & 3.50 & 1.40 & 0.20 & setosa \\ 
		2 & 4.90 & 3.00 & 1.40 & 0.20 & setosa \\ 
		3 & 4.70 & 3.20 & 1.30 & 0.20 & setosa \\ 
		4 & 4.60 & 3.10 & 1.50 & 0.20 & setosa \\ 
		5 & 5.00 & 3.60 & 1.40 & 0.20 & setosa \\ 
		6 & 5.40 & 3.90 & 1.70 & 0.40 & setosa \\ 
		\bottomrule
	\end{tabular}
\end{table}


\section{R代码}

\subsection{使用listings}
\begin{Rout}
	curve(dnorm(x), xlim=c(-4,4))
	curve(dnorm(x), xlim=c(-4,4))
	curve(dnorm(x), xlim=c(-4,4))
\end{Rout}

\section{Python代码}

% https://en.wikibooks.org/wiki/LaTeX/Source_Code_Listings

\subsection{使用listings}
\begin{pyout}[frame=single]
	import matplotlib.pyplot as plt
	import numpy as np
	x = np.arange(0.0, 6.0, 0.01)
	plt.plot(x, [x**2 for x in x])
	plt.show()
\end{pyout}



\subsection{使用pythonhighlight}
\begin{python}
	import matplotlib.pyplot as plt
	import numpy as np
	x = np.arange(0.0, 6.0, 0.01)
	plt.plot(x, [x**2 for x in x])
	plt.show()
\end{python}	

\end{appendix}



\end{document} 
